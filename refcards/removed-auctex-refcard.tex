% Reference Card for AUCTeX version 11.86
%**start of header
\newcount\columnsperpage

% This file has only been checked with 3 columns per page.  But it
% should print fine either via DVI or PDFTeX.

\columnsperpage=3

% Papersize stuff.  Use default paper size for PDF, but switch
% orientation.  Use papersize special for dvips.

\ifx\pdfoutput\undefined
  \csname newcount\endcsname\pdfoutput
  \pdfoutput=0
\fi

\ifnum\pdfoutput=0
%    \special{papersize 8.5in,11in}%
     \special{papersize 297mm,210mm}%
\else
      \dimen0\pdfpagewidth
      \pdfpagewidth\pdfpageheight
      \pdfpageheight\dimen0
\fi


% This file is intended to be processed by plain TeX (TeX82).
% compile-command: "tex tex-ref" or "pdftex tex-ref"
%
% Original author of Auc-TeX Reference Card:
%                                  
%       Terrence Brannon, PO Box 5027, Bethlehem, PA 18015 , USA
%  internet: tb06@pl118f.cc.lehigh.edu  (215) 758-1720 (215) 758-2104
%
% Kresten Krab Thorup updated the reference card to 6.
% Per Abrahamsen updated the reference card to 7, 8, and 9.
% Ralf Angeli updated it to 11.50.
% And David Kastrup messed around with it, too, merging the math reference.
%
% Thanks to Stephen Gildea
% Paul Rubin, Bob Chassell, Len Tower, and Richard Mlynarik
% for creating the GNU Emacs Reference Card from which this was mutated

\def\versionnumber{11.86}
\def\year{2010}
\def\version{February \year\ v\versionnumber}

\def\shortcopyrightnotice{\vskip 1ex plus 2 fill
  \centerline{\small \copyright\ \year\ Free Software Foundation, Inc.
  Permissions on back.  v\versionnumber}}

\def\copyrightnotice{%
\vskip 1ex plus 2 fill\begingroup\small
\centerline{Copyright \copyright\ 1987, 1992, 1993, 2004, 2005, 2008,}
\centerline{2010 Free Software Foundation, Inc.}
\centerline{for AUC\TeX\ version \versionnumber}

Permission is granted to make and distribute copies of
this card provided the copyright notice and this permission notice
are preserved on all copies.


\endgroup}

% make \bye not \outer so that the \def\bye in the \else clause below
% can be scanned without complaint.
\def\bye{\par\vfill\supereject\end}

\newdimen\intercolumnskip
\newbox\columna
\newbox\columnb

\edef\ncolumns{\the\columnsperpage}

\message{[\ncolumns\space 
  column\if 1\ncolumns\else s\fi\space per page]}

\def\scaledmag#1{ scaled \magstep #1}

% This multi-way format was designed by Stephen Gildea
% October 1986.
\if 1\ncolumns
  \hsize 4in
  \vsize 10in
  \voffset -.7in
  \font\titlefont=\fontname\tenbf \scaledmag3
  \font\headingfont=\fontname\tenbf \scaledmag2
  \font\smallfont=\fontname\sevenrm
  \font\smallsy=\fontname\sevensy

  \footline{\hss\folio}
  \def\makefootline{\baselineskip10pt\hsize6.5in\line{\the\footline}}
\else
  \hsize 3.2in
  \vsize 7.6in
  \hoffset -.75in
  \voffset -.8in
  \font\titlefont=cmbx10 \scaledmag2
  \font\headingfont=cmbx10 \scaledmag1
  \font\smallfont=cmr6
  \font\smallsy=cmsy6
  \font\eightrm=cmr8
  \font\eightbf=cmbx8
  \font\eightit=cmti8
  \font\eighttt=cmtt8
  \font\eightsl=cmsl8
  \font\eightsc=cmcsc8
  \font\eightsy=cmsy8
  \textfont0=\eightrm
  \textfont2=\eightsy
  \def\rm{\fam0 \eightrm}
  \def\bf{\eightbf}
  \def\it{\eightit}
  \def\tt{\eighttt}
  \def\sl{\eightsl}
  \def\sc{\eightsc}
  \normalbaselineskip=.8\normalbaselineskip
  \ht\strutbox.8\ht\strutbox
  \dp\strutbox.8\dp\strutbox
  \normallineskip=.8\normallineskip
  \normallineskiplimit=.8\normallineskiplimit
  \normalbaselines\rm           %make definitions take effect

  \if 2\ncolumns
    \let\maxcolumn=b
    \footline{\hss\rm\folio\hss}
    \def\makefootline{\vskip 2in \hsize=6.86in\line{\the\footline}}
  \else \if 3\ncolumns
    \let\maxcolumn=c
    \nopagenumbers
  \else
    \errhelp{You must set \columnsperpage equal to 1, 2, or 3.}
    \errmessage{Illegal number of columns per page}
  \fi\fi

  \intercolumnskip=.46in
  \def\abc{a}
  \output={%
      % This next line is useful when designing the layout.
      %\immediate\write16{Column \folio\abc\space starts with \firstmark}
      \if \maxcolumn\abc \multicolumnformat \global\def\abc{a}
      \else\if a\abc
        \global\setbox\columna\columnbox \global\def\abc{b}
        %% in case we never use \columnb (two-column mode)
        \global\setbox\columnb\hbox to -\intercolumnskip{}
      \else
        \global\setbox\columnb\columnbox \global\def\abc{c}\fi\fi}
  \def\multicolumnformat{\shipout\vbox{\makeheadline
      \hbox{\box\columna\hskip\intercolumnskip
        \box\columnb\hskip\intercolumnskip\columnbox}
      \makefootline}\advancepageno}
  \def\columnbox{\leftline{\pagebody}}

  \def\bye{\par\vfill\supereject
    \if a\abc \else\null\vfill\eject\fi
    \if a\abc \else\null\vfill\eject\fi
    \end}  
\fi

% we won't be using math mode much, so redefine some of the characters
% we might want to talk about
\catcode`\^=12
\catcode`\_=12

\chardef\\=`\\
\chardef\{=`\{
\chardef\}=`\}

\hyphenation{mini-buf-fer}

\parindent 0pt
\parskip 1ex plus .5ex minus .5ex

\def\small{\smallfont\textfont2=\smallsy\baselineskip=.8\baselineskip}

\def\newcolumn{\vfill\eject}

\def\title#1{{\titlefont\centerline{#1}}\vskip 1ex plus .5ex}

\def\section#1{\par\vskip 0pt plus 0.2\vsize \penalty-3000
         \vskip 0pt plus -0.2\vsize
  \vskip 3ex plus 2ex minus 2ex {\headingfont #1}\mark{#1}%
  \vskip 2ex plus 1ex minus 1.5ex}

\newdimen\keyindent

\def\beginindentedkeys{\keyindent=1em}
\def\endindentedkeys{\keyindent=0em}
\endindentedkeys

\def\paralign{\vskip\parskip\halign}

\def\<#1>{$\langle${\rm #1}$\rangle$}

\def\kbd#1{{\tt#1}\null}        %\null so not an abbrev even if period follows

\def\beginexample{\par\leavevmode\begingroup
  \obeylines\obeyspaces\parskip0pt\tt}
{\obeyspaces\global\let =\ }
\def\endexample{\endgroup}

\def\key#1#2{\leavevmode\hbox to \hsize{\vtop
  {\hsize=.68\hsize\rightskip=1em
  \hskip\keyindent\relax#1}\kbd{#2}\hfil}}

\newbox\metaxbox
\setbox\metaxbox\hbox{\kbd{M-x }}
\newdimen\metaxwidth
\metaxwidth=\wd\metaxbox

\def\metax#1#2{\leavevmode\hbox to \hsize{\hbox to .75\hsize
  {\hskip\keyindent\relax#1\hfil}%
  \hskip -\metaxwidth minus 1fil
  \kbd{#2}\hfil}}

\def\threecol#1#2#3{\hskip\keyindent\relax#1\hfil&\kbd{#2}\quad
  &\kbd{#3}\quad\cr}

\def\LaTeX{%
    L\kern-.36em\raise.3ex\hbox{\sc{a}}\kern-.15em\TeX}

%**end of header

\title{AUC\TeX\ Reference Card}

\centerline{(for version \versionnumber)}

\newcolumn

\iftrue % RefTeX long version

\title{RefTeX}

\fi

\bye

%%% Local Variables: 
%%% mode: plain-TeX
%%% TeX-master: t
%%% End: 
